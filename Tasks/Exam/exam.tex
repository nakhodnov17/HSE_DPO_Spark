\documentclass[10pt,fleqn]{article}

\usepackage[russian]{babel}
\usepackage[utf8]{inputenc}
\usepackage{color}
\usepackage{amsmath}
\usepackage{amssymb}
\usepackage{graphics}
\usepackage{epsfig}
\usepackage{bm}
\usepackage[colorlinks,urlcolor=blue]{hyperref}
\usepackage{tikz}
\usepackage{pgfplots}
\usepackage{verbatim}
\usepackage{mdframed}
\usepackage{dirtree}
\usepackage{indentfirst}
\usepackage{url}

\definecolor{codegray}{gray}{0.9}
\newcommand{\code}[1]{%
  \begingroup\setlength{\fboxsep}{1pt}%
  \colorbox{codegray}{\texttt{\hspace*{2pt}\vphantom{Ay}#1\hspace*{2pt}}}%
  \endgroup
}
% mdinlinecode command for including code snippets inline
% (fake verbatim, so all special character should be escaped,
% or textmode equivalents of special characters should be used)
\definecolor{mdinlinecodeboxframecolor}{HTML}{DDDDDD}
\definecolor{mdinlinecodeboxbackgroundcolor}{HTML}{F8F8F8}
\newcommand{\mdinlinecode}[1]{%
    \begin{tikzpicture}[baseline=0ex]%
        \node[anchor=base,%
            text height=0.9em,%
            text depth=0.9ex,%
            inner ysep=0pt,%
            draw=mdinlinecodeboxframecolor,%
            fill=mdinlinecodeboxbackgroundcolor,%
            rounded corners=1.5pt] at (0,0) {\small\texttt{#1}};%
    \end{tikzpicture}%
}

\newmdenv[font=\footnotesize,%
linewidth=0.4pt,%
roundcorner=2pt,%
linecolor=mdinlinecodeboxframecolor,%
backgroundcolor=mdinlinecodeboxbackgroundcolor,%
settings={\setlength{\parindent}{0pt}}]{mdcdblk}
\newenvironment{mdcodeblock}{\endgraf\verbatim}{\endverbatim}
\BeforeBeginEnvironment{mdcodeblock}{\begin{mdcdblk}}
\AfterEndEnvironment{mdcodeblock}{\end{mdcdblk}}

\textheight=26cm % высота текста
\textwidth=18cm % ширина текста
\oddsidemargin=-1cm % отступ от левого края
\topmargin=-3cm % отступ от верхнего края
\sloppy

\newcounter{example}

%-- Обозначение вектора жирным символом
\def\vec#1{\mathchoice{\mbox{\boldmath$\displaystyle#1$}}
{\mbox{\boldmath$\textstyle#1$}} {\mbox{\boldmath$\scriptstyle#1$}} {\mbox{\boldmath$\scriptscriptstyle#1$}}}

\DeclareMathOperator{\B}{Bin}
\DeclareMathOperator{\Ps}{Poiss}
\DeclareMathOperator{\R}{Unif}
\DeclareMathOperator{\sign}{\mathrm{sign}}
\DeclareMathOperator{\softmax}{\mathrm{softmax}}
\DeclareMathOperator{\loss}{\mathcal{L}}

\pagestyle{empty}


\begin{document}

\begin{center}
    \begin{tabular}{|p{17.5cm}|}
        \hline
        \textbf{Высшая Школа Экономики. ФКН}\\
        \begin{center} \Large Контрольная работа. \end{center}\\
        \textbf{"DS-09. Промышленное машинное обучение на Spark"}\\
        \hline
    \end{tabular}
\end{center}

\

\begin{tabbing}
    Начало выполнения задания: 19 апреля 2023 года, 17:00.\\
    Жёсткий Дедлайн: \textcolor{red}{\bf 04 мая 2023 года, 23:59.}
\end{tabbing}

\begin{abstract}
    Вам предлагается решить $10$ задач с использованием Spark. Описание каждой задачи и формат входных данных указаны ниже. Решения задач необходимо оформить в виде Jupyter ноутбука. Задача считается решённой при выполнении следующих условий:
    \begin{itemize}
        \item Приведён правильный код для решения задачи
        \item Продемонстрирована корректность работы кода на примере из условия
        \item Было придумано два нетривиальных и различных примера входных данных и продемонстрирована корректность работы кода на данных примерах 
        \item Примеры оформлены в виде \mdinlinecode{pyspark.sql.DataFrame}
        \item Решение использует только один \textbf{action} — \mdinlinecode{.toPandas} для отображения финального результата
        \item Решение использует только библиотеку \mdinlinecode{pyspark}
        \item В решении не используются SQL запросы (например, метод \mdinlinecode{.sql} или \mdinlinecode{.select} с SQL выражением) кроме как для операции \textbf{unpivot}
    \end{itemize}

    Порядок строк в ответе не важен, если явно не указано иное.

    В качестве подсказки для каждой задачи указана ссылка на аналогичную задачу с решением через SQL запросы.

    Решение в виде ноутбука с названием \mdinlinecode{[Exam][ФИО].ipynb} нужно отправить на почту \href{nakhodnovms@my.msu.ru}{nakhodnovms@my.msu.ru} с темой письма \mdinlinecode{[HSE Spark 2022][Контрольная Работа][ФИО]}.
\end{abstract}

\tableofcontents

\newpage


\begin{section}{\href{https://leetcode.com/problems/classes-more-than-5-students/}{Classes More Than 5 Students}}

Есть таблица курсов с колонками: студент и предмет. Перечислите все предметы, которым обучается не менее $5$ человек.

\begin{figure}[!h]
    \centering
    \begin{tabular}{ccc}
        \textbf{INPUT} & & \textbf{OUTPUT} \\
        & & \\
        \begin{tabular}{|c|c|}
    \hline
        \textbf{student} & \textbf{class} \\ \hline
        A & Math \\ 
        B & English \\ 
        C & Math \\ 
        D & Biology \\ 
        E & Math \\ 
        F & Computer \\ 
        G & Math \\ 
        H & Math \\ 
        I & Math \\ \hline
\end{tabular}
 & $\Longrightarrow$ & \begin{tabular}{|c|}
    \hline
        \textbf{class} \\ \hline
        Math \\ \hline
\end{tabular}
 \\
     \end{tabular}
     \caption{Пример к задаче 1. \label{fig:task_01}}
\end{figure}

\end{section}


\begin{section}{\href{https://leetcode.com/problems/second-highest-salary/}{Second Highest Salary}}

Определите значение второй по величине зарплаты из таблицы с зарплатами сотрудников. Если второй по величине заработной платы нет, запрос
должен вернуть строку \mdinlinecode{"Missing"}.

\begin{figure}[!h]
    \centering
    \begin{tabular}{ccc}
        \textbf{INPUT} & & \textbf{OUTPUT} \\
        & & \\
        \begin{tabular}{|c|c|}
    \hline
        \textbf{id} & \textbf{salary} \\ \hline
        1 & 100 \\ 
        2 & 200 \\ 
        3 & 300 \\ \hline
\end{tabular}
 & $\Longrightarrow$ & \begin{tabular}{|c|}
\hline
    \textbf{SecondHighestSalary}\\ \hline
    200 \\ \hline
\end{tabular}
 \\
        & & \\
        \begin{tabular}{|c|c|}
    \hline
        \textbf{id} & \textbf{salary} \\ \hline
        1 & 100 \\ \hline
\end{tabular}
 & $\Longrightarrow$ & \begin{tabular}{|c|}
\hline
    \textbf{SecondHighestSalary}\\ \hline
    Missing \\ \hline
\end{tabular}
 \\
     \end{tabular}
     \caption{Пример к задаче 2. \label{fig:task_02}}
\end{figure}

\end{section}


\begin{section}{\href{https://leetcode.com/problems/rising-temperature/}{Rising Temperature}}

В таблице содержится информация о температуре в определенный день. Найдите идентификаторы всех дат с более высокой температурой по сравнению с предыдущим ("вчерашним") днём.

\begin{figure}[!h]
    \centering
    \begin{tabular}{ccc}
        \textbf{INPUT} & & \textbf{OUTPUT} \\
        & & \\
        \begin{tabular}{|c|c|c|}
    \hline
        \textbf{id} & \textbf{recordDate} & \textbf{temperature} \\ \hline
        1 & 2015-01-01 & 10 \\ 
        2 & 2015-01-02 & 25 \\ 
        3 & 2015-01-03 & 20 \\ 
        4 & 2015-01-04 & 30 \\ \hline
\end{tabular}
     & $\Longrightarrow$ & \begin{tabular}{|c|c|}
    \hline
        \textbf{id} \\ \hline
        2 \\ 
        4 \\ \hline
\end{tabular} \\
     \end{tabular}
     \caption{Пример к задаче 3. \label{fig:task_03}}
\end{figure}

\end{section}

\newpage

\begin{section}{\href{https://leetcode.com/problems/duplicate-emails/}{Duplicate Emails}}

Удалите из таблицы с адресами электронной почты повторяющиеся адреса, оставив только уникальные, которые соответствуют наименьшему идентификатору.

    \begin{figure}[!h]
        \centering
        \begin{tabular}{ccc}
            \textbf{INPUT} & & \textbf{OUTPUT} \\
            & & \\
            \begin{tabular}{|c|c|}
    \hline
        \textbf{id} & \textbf{email} \\ \hline
        1 & john@example.com \\ 
        2 & bob@example.com \\ 
        3 & john@example.com \\ \hline
\end{tabular} & $\Longrightarrow$ & \begin{tabular}{|c|c|}
    \hline
        \textbf{id} & \textbf{email} \\ \hline
        1 & john@example.com \\ 
        2 & bob@example.com \\ \hline
\end{tabular} \\
         \end{tabular}
         \caption{Пример к задаче 4. \label{fig:task_04}}
    \end{figure}

\end{section}


\begin{section}{\href{https://leetcode.com/problems/customers-who-never-order/}{Customers Who Never Order}}

По таблицам с информацией о клиентах и их заказах определите имена клиентов, которые ничего не заказывали.

\begin{figure}[!h]
    \centering
    \begin{tabular}{ccc}
        \textbf{INPUT} & & \textbf{OUTPUT} \\
        & & \\
        \begin{tabular}{|c|c|}
    \hline
        \textbf{id} & \textbf{name} \\ \hline
        1 & Joe \\ 
        2 & Henry \\ 
        3 & Sam \\ 
        4 & Max \\ \hline
\end{tabular}
 &  &  \\
         & $\Longrightarrow$ & \begin{tabular}{|c|c|}
    \hline
        \textbf{Customers} \\ \hline
        Henry \\ 
        Max  \\ \hline
\end{tabular} \\
        \begin{tabular}{|c|c|}
    \hline
        \textbf{id} & \textbf{customerId} \\ \hline
        1 & 3 \\ 
        2 & 1 \\ \hline
\end{tabular}
 &  &  \\
     \end{tabular}
     \caption{Пример к задаче 5. \label{fig:task_05}}
\end{figure}

\end{section}


\begin{section}{\href{https://leetcode.com/problems/employees-earning-more-than-their-managers/}{Employees Earning More Than Their Managers}}

Таблица содержит информацию о всех сотрудниках, включая их руководителей. У каждого сотрудника есть идентификатор, а также столбец с идентификатором их руководителя. Определите всех сотрудников, которые
зарабатывают больше, чем их руководитель.

    \begin{figure}[!h]
        \centering
        \begin{tabular}{ccc}
            \textbf{INPUT} & & \textbf{OUTPUT} \\
            & & \\
            \begin{tabular}{|c|c|c|c|}
    \hline
        \textbf{id} & \textbf{name} & \textbf{salary} & \textbf{managerId} \\ \hline
        1 & Joe & 70000 & 3 \\ 
        2 & Henry & 80000 & 4 \\ 
        3 & Sam & 60000 & Null \\ 
        4 & Max & 90000 & Null \\ \hline
\end{tabular}
 & $\Longrightarrow$ & \begin{tabular}{|c|}
    \hline
        \textbf{Employee} \\ \hline
        Joe \\ \hline
\end{tabular} \\
         \end{tabular}
         \caption{Пример к задаче 6. \label{fig:task_06}}
    \end{figure}
    
\end{section}

\newpage
\begin{section}{\href{https://leetcode.com/problems/combine-two-tables/}{Combine Two Tables}}

Объедините две таблицы так, чтобы определить поля \mdinlinecode{FirstName, LastName, City, State} для каждого человека в первой входной таблице.

    \begin{figure}[!htb]
        \centering
        \begin{tabular}{ccc}
            \textbf{INPUT} & & \textbf{OUTPUT} \\
            & & \\
            \begin{tabular}{|c|c|c|}
    \hline
        \textbf{personId} & \textbf{lastName} & \textbf{firstName} \\ \hline
        1 & Wang & Allen \\ 
        2 & Alice & Bob \\ \hline
\end{tabular}
 &  &  \\
             & $\Longrightarrow$ & \begin{tabular}{|c|c|c|c|}
    \hline
        \textbf{firstName} & \textbf{lastName} & \textbf{city} & \textbf{state} \\ \hline
        Allen & Wang & Null & Null \\ 
        Bob & Alice & New York City & New York \\ \hline
\end{tabular}
 \\
            \begin{tabular}{|c|c|c|c|}
    \hline
        \textbf{addressId} & \textbf{personId} & \textbf{city} & \textbf{state} \\ \hline
        1 & 2 & New York City & New York \\ 
        2 & 3 & Leetcode & California \\ \hline
\end{tabular}
 &  &  \\
         \end{tabular}
         \caption{Пример к задаче 7. \label{fig:task_07}}
    \end{figure}

\end{section}


\begin{section}{\href{https://leetcode.com/problems/not-boring-movies/}{Not Boring Movies}}
    В городе $X$ открылся новый кинотеатр. Для каждого фильма в прокате известно его описание и рейтинг. Определите все фильмы с нечётными идентификаторами и описанием, которое не содержит подстроку \mdinlinecode{"boring"}. Выведите результат, отсортировав получившиеся фильмы по рейтингу. 

    \begin{figure}[!h]
        \centering
        \begin{tabular}{ccc}
            \textbf{INPUT} & & \textbf{OUTPUT} \\
            & & \\
            \begin{tabular}{|c|c|c|c|}
    \hline
        \textbf{id} & \textbf{movie} & \textbf{description} & \textbf{rating} \\ \hline
        1 & War & great 3D & 8.9 \\ 
        2 & Science & fiction & 8.5 \\ 
        3 & irish & boring & 6.2 \\ 
        4 & Ice song & Fantacy & 8.6 \\ 
        5 & House card & Interesting & 9.1 \\ \hline
\end{tabular}
 & $\Longrightarrow$ & \begin{tabular}{|c|c|c|c|}
    \hline
        \textbf{id} & \textbf{movie} & \textbf{description} & \textbf{rating} \\ \hline
        5 & House card & Interesting & 9.1 \\ 
        1 & War & great 3D & 8.9 \\ \hline
\end{tabular}
 \\
         \end{tabular}
         \caption{Пример к задаче 8. \label{fig:task_08}}
    \end{figure}
\end{section}


\begin{section}{\href{https://leetcode.com/problems/swap-salary/}{Swap Salary}}

    Преобразуйте исходную таблицу, поменяв во всех записях пол сотрудников на противоположный ($\text{m} \iff \text{f}$).

    \begin{figure}[!h]
        \centering
        \begin{tabular}{ccc}
            \textbf{INPUT} & & \textbf{OUTPUT} \\
            & & \\
            \begin{tabular}{|c|c|c|c|}
    \hline
        \textbf{id} & \textbf{name} & \textbf{sex} & \textbf{salary} \\ \hline
        1 & A & m & 2500 \\ 
        2 & B & f & 1500 \\ 
        3 & C & m & 5500 \\ 
        4 & D & f & 500 \\ \hline
\end{tabular}
 & $\Longrightarrow$ & \begin{tabular}{|c|c|c|c|}
    \hline
        \textbf{id} & \textbf{name} & \textbf{sex} & \textbf{salary} \\ \hline
        1 & A & f & 2500 \\ 
        2 & B & m & 1500 \\ 
        3 & C & f & 5500 \\ 
        4 & D & m & 500 \\ \hline
\end{tabular}
 \\
         \end{tabular}
         \caption{Пример к задаче 9. \label{fig:task_09}}
    \end{figure}
\end{section}

\newpage
\begin{section}{\href{https://leetcode.com/problems/big-countries/}{Big Countries}}

Будем называть страну \textit{великой}, если ее площадь превышает $3$ миллиона квадратных километров или население превышает $25$ миллионов.

Выведите названия, население и площадь для всех \textit{великих} стран из входной таблицы.

    \begin{figure}[!h]
        \centering
        \begin{tabular}{ccc}
            \textbf{INPUT} & & \textbf{OUTPUT} \\
            & & \\
            \begin{tabular}{|c|c|c|c|c|}
    \hline
        \textbf{name} & \textbf{continent} & \textbf{area} & \textbf{population} & \textbf{gdp} \\ \hline
        Afghanistan & Asia & 652230 & 25500100 & 20343000000 \\ 
        Albania & Europe & 28748 & 2831741 & 12960000000 \\ 
        Algeria & Africa & 2381741 & 37100000 & 188681000000 \\ 
        Andorra & Europe & 468 & 78115 & 3712000000 \\ 
        Angola & Africa & 1246700 & 20609294 & 100990000000 \\ \hline
\end{tabular}
 & $\Longrightarrow$ & \begin{tabular}{|c|c|c|}
    \hline
        \textbf{name} & \textbf{population} & \textbf{area} \\ \hline
        Afghanistan & 25500100 & 652230 \\ 
        Algeria & 37100000 & 2381741 \\ \hline
\end{tabular}
 \\
         \end{tabular}
         \caption{Пример к задаче 10. \label{fig:task_10}}
    \end{figure}
\end{section}


\end{document}